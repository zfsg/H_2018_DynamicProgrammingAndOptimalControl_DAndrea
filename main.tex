
\documentclass[10pt,a4paper]{scrartcl}

\usepackage[english]{babel}

\input{../Headerfiles/Packages}
\input{../Headerfiles/Titles}
\input{../Headerfiles/Commands}
\graphicspath{{Pictures/}}
\parindent 0pt

\title{Dynamic Programming And Optimal Control}
\author{GianAndrea Müller}

\newtheorem{define}{Definition}
\newtheorem{theorem}{Theorem}

\begin{document}
\begin{multicols*}{4}
\maketitle
\tableofcontents
\end{multicols*}

\begin{multicols*}{2}
\section{Random Variables}

\subsection{Discrete Random Variables (DRV)}

\begin{TDefinitionTable*}
$\mathcal{X}$&set of all possible outcomes\\
$p_x(\cdot)$&probability density function (PDF)\\
\end{TDefinitionTable*}

\begin{enumerate}
\item $p_x(\bar{x})\geq 0\ \forall\bar{x}\in\mathcal{X}$
\item $\sum\limits_{\bar{x}\in\mathcal{X}}p_x(\bar{x})=1$
\end{enumerate}

\begin{define}
$p_x(\cdot)$ and $\mathcal{X}$ define a \textbf{discrete random variables} (DRV) $x$.
\end{define}

The probability that a random variable $x$ is equal to some value $\bar{x}\in\mathcal{X}$ is $p_x(\bar{x})$. This is written as $Pr(x=\bar{x})=p_x(\bar{x})$.

\begin{define}
The \textbf{joint PDF} $p_{xy}(\cdot,\cdot)$ is a real valued function that satisfies:

\begin{enumerate}
\item $p_{xy}(\bar{x},\bar{y})\geq 0\ \forall\bar{x}\in\mathcal{X},\ \forall y\in\mathcal{Y}$,
\item $\sum\limits_{\bar{x}\in\mathcal{X}}\sum\limits_{y\in\mathcal{Y}}p_{xy}(\bar{x},\bar{y})=1$.
\end{enumerate}
\end{define}

\begin{define}
\textbf{Marginalization} or \textbf{Sum Rule} axiom:

Given $p_{xy}(\cdot,\cdot)$ define $p_x(\bar{x}):=\sum\limits_{\bar{y}\in\mathcal{Y}}p_{xy}(\bar{x},\bar{y})$.
\end{define}

\begin{define}
\textbf{Conditioning} or \textbf{Product Rule} axiom:

Given $p_{xy}(\cdot,\cdot)$ the PDF of $y$ is $p_{x|y}(\bar{x}|\bar{y}):=\frac{p_{xy}(\bar{x},\bar{y})}{p_y(\bar{y})}$ when $p_y(\bar{y})\neq 0$.
\end{define}

\begin{itemize}
\item Sum rule applied to a conditional PDF:

Given $p_{xy|z}(\bar{x},\bar{y}|\bar{z}),p_{x|z}(\bar{x}|\bar{z}:=\sum\limits_{\bar{y}\in\mathcal{Y}}p_{xy|z}(\bar{x},\bar{y}|\bar{z})$.
\item Short form: $p(x|y)$
\item Product rule usually written as: $p(x,y) = p(x|y)p(y) =p(y|x)p(x)$
\end{itemize}

\begin{define}
\textbf{Total Probability Theorem:}

\mportant{$p_x(\bar{x})=\sum\limits_{\bar{y}\in\mathcal{Y}}p_{x|y}(\bar{x}|\bar{y})p_y(\bar{y})$}.
\end{define}

\subsubsection{Generalization for multiple variables}

\begin{define}
\textbf{Marginalization}

\important{$p_x(\bar{x})=\sum\limits_{\bar{y}\in\mathcal{Y}}p_{xy}(\bar{x},\bar{y})$}

as a short form of:

\mportant{$p_{x_1,\ldots,x_N}(\bar{x}_1,\ldots,\bar{x}_N)=\sum\limits_{(\bar{y}_1,\ldots,\bar{y}_L)\in\mathcal{Y}}p_{x_1,\ldots,x_N}(\bar{x}_1,\ldots,\bar{x}_N,\bar{y}_1,\ldots,\bar{y}_L)$.}
\end{define}

\begin{define}
\textbf{Conditioning}

\important{$p(x,y) = p(x|y)p(y)$}

as a short form of:

\mportant{$p(x_1,\ldots,x_N,y_1,\ldots,y_L)=p(x_1,\ldots,x_N|y_1,\ldots,y_L)p(y_1,\ldots,y_L)$}
\end{define}

\begin{define}
Random variables $x$ and $y$ are said to be \textbf{independent} if $p(x|y)=p(x)$. Equivalently: $p(x,y)=p(x)p(y)$.
\end{define}

\begin{define}
Random variables are said to be \textbf{conditionally independent} if: $p(x|y,z)=p(x|z)$. Knowledge of $z$ makes $x$ and $y$ independent.
\end{define}

\subsection{Continuous Random Variables (CRV)}

\begin{TDefinitionTable*}
$\mathcal{X}$&subset of the real line\\
$p(\cdot)$&PDF
\end{TDefinitionTable*}

\begin{enumerate}
\item $p_x(\bar{x})\geq 0\ \forall\bar{x}\in\mathcal{X}$
\item $\int_{\mathcal{X}}p_x(\bar{x})d\bar{x}=1$
\end{enumerate}

\begin{define}
The \textbf{probability of being in an interval} is:

\mportant{$Pr(x\in[a,b]):=\int_a^bp_x(\bar{x})d\bar{x}$}
\end{define}

\subsection{Expectation}

\begin{define}
The \textbf{expected value} of a random variable is defined as:

\mportant{$\underset{x}{E}[x]:=\sum\limits_{\bar{x}\in\mathcal{X}}\bar{x}p_x(\bar{x})$}
\end{define}

\begin{itemize}
\item $\underset{x}{E}[ax+b]=a\underset{x}{E}[x]+b$ where $a,b$ constant
\item $\underset{x}{E}[g(x)]=\sum\limits_{\bar{x}\in\mathcal{X}}\bar{x}p_{x|y}(\bar{x}|\bar{y})$.
\item For conditional PDF's:
\end{itemize}

\mportant{$\underset{x|y}{E}[x|y=\bar{y}]:=\sum\limits_{\bar{x}\in\mathcal{X}}\bar{x}p_{x|y}(\bar{x}|\bar{y})$}

\subsubsection{Multi-variable generalizations}

If $x$ is a vector:

\mportant{$\underset{x}{E}[x]=\sum\limits_{\bar{x}\in\mathcal{X}}\bar{x}p_x(\bar{x})=\sum\limits_{\bar{x_1}\in\mathcal{X}}\cdots\sum\limits_{\bar{x}_N\in\mathcal{X}}[\bar{x}_1,\ldots,\bar{x}_N]^Tp_{(x_1,\ldots,x_N)}(\bar{x}_1,\ldots,\bar{x}_N]$}

Given $g(x):\mathbb{R}^N\rightarrow \mathbb{R}$ and DRV $x$

\mportant{$\underset{x}{E}[g(x)]=\sum\limits_{\bar{x}\in\mathcal{X}}g(\bar{x})p_x(\bar{x})=\sum\limits_{\bar{x}_1\in\mathcal{X}}\cdots\sum\limits_{\bar{x}_N\in\mathcal{X}}g(\bar{x}_1,\ldots,\bar{x}_N)p_{(x_1,\ldots,x_N)}(\bar{x}_1,\ldots,\bar{x}_N)$}

If the two random variables are \textbf{independent}, then:

\mportant{$\underset{xy}{E}[g(x,y)]=\sum\limits_{\bar{y}\in\mathcal{Y}}\sum\limits_{\bar{X}\in\mathcal{X}}g(\bar{x},\bar{y})p_{xy}(\bar{x},\bar{y})=\sum\limits_{\bar{y}\in\mathcal{Y}}\sum\limits_{\bar{x}\in\mathcal{X}}g(\bar{x},\bar{y})p_{xy}(\bar{x},\bar{y})=\sum\limits_{\bar{y}\in\mathcal{Y}}\sum\limits_{\bar{x}\in\mathcal{X}}g(\bar{x},\bar{y})p_x(\bar{x})p_y(\bar{y})=\underset{y}{E}[\underset{x}{E}[g(x,y)]]$}

\textbf{Mean and Variance}:

\begin{define}
$\underset{x}{E}[x]$ is called the \textbf{mean}, generally a vector.
\end{define}

\begin{define}
$\underset{x}{\text{Var}}[x]:=\underset{x}{E}\left[\left(x-\underset{x}{E}[x]\right)\left(x-\underset{x}{E}[x]\right)^T\right]$ is called the \textbf{variance}, generally a matrix.
\end{define}

\textbf{Linerarity:}

\mportant{$\underset{xy}{E}[x+y]=\sum\limits_{\bar{y}\in\mathcal{Y}}\sum\limits_{\bar{x}\in\mathcal{X}}(\bar{x}+\bar{y})p_{xy}(\bar{x},\bar{y})=\sum\limits_{x\in\mathcal{x}}\bar{x}\sum\limits_{\bar{y}\in\mathcal{Y}}p_{xy}(\bar{x},\bar{y})+\sum\limits_{\bar{y}\in\mathcal{Y}}\bar{y}\sum\limits_{\bar{x}\in\mathcal{X}}p_{xy}(\bar{x},\bar{y})=\underset{x}{E}[x]+\underset{y}{E}[y]$}

\textbf{Law of Total Expectation:}

\mportant{$\underset{y}{E}[\underset{x|y}{E}[x]]=\sum\limits_{\bar{y}\in\mathcal{Y}}p_y(\bar{y})\left(\sum\limits_{\bar{x}\in\mathcal{X}}\bar{x}p_{x|y}(\bar{x}|\bar{y})\right)=\sum\limits_{\bar{x}\in\mathcal{X}}\bar{x}\sum\limits_{\bar{y}\in\mathcal{Y}}p_{x|y}(\bar{x}|\bar{y})p_y(\bar{y})=\sum\limits_{\bar{x}\in\mathcal{X}}\bar{x}p_x(\bar{x})=\underset{x}{E}[x]$}

\section{Basics}

\myspic{0.7}{Dynamics}

\mportant{$x_{k+1}=f_k(x_k,u_k,w_k),\ k=0,1,\ldots,N-1$}

\begin{TDefinitionTable*}
$k$&discrete time index&\\
$N$&given time horizon&\\
$x_k\in \mathcal{S}_k$&system state vector at time $k$&\\
$u_k\in\mathcal{U}_k(x_k)$&control input vector at time $k$&\\
$\omega_k$&disturbance vector at time $k$&\\
$f_k(\cdot,\cdot,\cdot)$&function capturing system evolution at time $k$
\end{TDefinitionTable*}

\begin{itemize}
\item It is assumed that the conditional probability of $w_k$ depending on $u_k$ and $x_k$ is known and $w_k$ is independent of any other variables.
\end{itemize}

\subsection{Cost Function}

\important{$\underbrace{g_N(x_N)}_{\text{terminal cost}}+\underbrace{\sum\limits_{k=0}^{N-1}\underbrace{g_k(x_k,u_k,w_k)}_{\text{stage cost}}}_{\text{accumulated cost}}$}

\begin{itemize}
\item Cost is a random variable.
\end{itemize}

\subsubsection{Expected Cost}

\begin{TDefinitionTable*}
$X_1:=(x_1,\ldots,x_N)$&set of all states&\\
$U_0:=(u_0,\ldots,u_{N-1}$&set of all inputs&\\
$W_0:=(w_0,\ldots,w_{N-1}$&set of disturbances&\\
\end{TDefinitionTable*}

\important{$\underset{(X_1,U_0,W_0|x_0)}{E}\left[g_N(x_N)+\sum\limits_{k=0}^{N-1}g_k(x_k,u_k,w_k)\right]$}

\begin{itemize}
\item All variables $x_k,u_k,w_k$ are in general all random variables, since at least the disturbance is random, and thus coupled by the dynamics and possibly a control law depending on the state, all variables are coupled in general.
\end{itemize}

\subsection{Open Loop Control}

\begin{TDefinitionTable*}
$\bar{U}_0:=(\bar{u}_0,\ldots,\bar{u}_{N-1})$&fixed set of control inputs&\\
\end{TDefinitionTable*}

\mportname{Open Loop Cost}{$g_N(x_N)+\sum\limits_{k=0}^{N-1}g_k(x_k,\bar{u}_k,w_k)$}

\importname{Expected Open Loop Cost}{$\underset{(X_1,W_0|x_0)}{E}\left[g_N(x_N)+\sum\limits_{k=}^{N-1}g_k(x_k,\bar{u}_k,w_k)\right]$}

\begin{itemize}
\item The bar emphasizes that the control inputs are fixed.
\item Open loop control means that the control law is defined and fixed at time zero. The measurements of the state are not used to adapt the control law.
\end{itemize}

\subsection{Closed Loop Control}

\begin{TDefinitionTable*}
$u_k=\mu_k(x),\ u_k\in\mathcal{U}_k(x),\ \forall x\in\mathcal{S}_k,\ k=0,\ldots,N-1$&control law&\\
\end{TDefinitionTable*}

\importname{Admissible Policy}{$\pi:=(\mu_0(\cdot),\mu_1(\cdot),\ldots,\mu_{N-1}(\cdot))$}

\mportname{Closed Loop Cost}{$g_N(x_N)+\sum\limits_{k=0}^{N-1}g_k(x_k,\mu_k(x_k),w_k)$}

\importname{Expected Closed Loop Cost}{$J_\pi(x):=\underset{(X_1,W_0|x_0=x)}{E}\left[g_N(x_N)+\sum\limits_{k=0}^{N-1}g_k(x_k,\mu_k(x_k),w_k)\right]$}

\begin{TDefinitionTable*}
$\Pi$&set of all admissible policies&\\
\end{TDefinitionTable*}

$\pi^\ast$ is called the optimal policy if

\mportant{$J_{\pi^\ast}(x)\leq J_{\pi}(x)\quad\forall\pi\ \in\ \pi,\ \forall x\ \in\ \mathcal{S}_0$}

\subsection{Discrete State and Finite State Problems}

\begin{TDefinitionTable*}
$P_{ij}(u,k):=p_{x_{k+1}|x_k,u_k}(j|i,u)=Pr(x_{k+1}=j|x_k=i,u_k=u)$&transition probability&\\
\end{TDefinitionTable*}

\begin{itemize}
\item $p_{x_{k+1}|x_k,u_k}(\cdot|\cdot,\cdot)$ denotes the PDF of $x_{k+1}$ given $x_k$ and $u_k$.
\end{itemize}

\section{The Dynamic Programming Algorithm}

\subsection{The standard problem formulation}

\mportant{$x_{k+1}=f_k(x_k,u_k,w_k),\quad k=0,1,2,\ldots,N-1$}

where $x_k\in\mathcal{S}_k,\ u_k\in\mathcal{U}_k$ and $w_k\sim p_{w_k|x_k,u_k}$

The control inputs are generated by the admissible policy $\pi\in\Pi$

\mportant{$\pi=(\mu_0(\cdot),\mu_1(\cdot),\ldots,\mu_{N-1}(\cdot))$}

The expected closed loop cost, given $x\in\mathcal{S}_0$, associated with policy $\pi$ is:

\mportant{$J_\pi(x)=\underset{(X_1,W_0|x_0=x)}{E}\left[g_N(x_N)+\sum\limits_{k=0}^{N-1}g_k(x_k,\mu_k(x_k),w_k)\right]$}

Now the objective is to construct $\pi^\ast$, an optimal policy such that

\mportant{$\pi^\ast=\underset{\pi\in\Pi}{\text{arg min}}J_\pi(x)$.}

$\pi^\ast$ is not a function of the state, thus $\pi^\ast$ has to work for all possible $x$.

\subsection{Principle of Optimality}

Let $\pi^\ast=(\mu_0^\ast(\cdot),\mu_1^\ast(\cdot),\ldots,\mu_{N-1}^\ast(\cdot))$ be an optimal policy.

\mportant{$\underset{(X_{i+1},W_i|x_i=x)}{E}\left[g_N(x_N)+\sum\limits_{k=i}^{N-1}g_k(x_k,\mu_k(x_k),w_k)\right]$}

where $X_{i+1}:=(x_{i+1},\ldots,x_N)$ and $W_i:=(w_i,\ldots,w_{N-1})$. Then the \textbf{truncated} policy $(\mu_i^\ast(\cdot),\mu_{i+1}^\ast(\cdot),\ldots,\mu_{N-1}^\ast(\cdot))$ is optimal for this problem.

\subsection{DPA}

\begin{theorem}
For any initial state $x\in\mathcal{S}_0$, the optimal cost $J^\ast(x)$ is equal to $J_0(x)$ given the following recursive algorithm:

\textbf{Initialization}

\mportant{$J_N(x)=g_N(x),\quad \forall x\in\mathcal{S}_N$}

\textbf{Recursion}

\mportant{$J_k(x):=\min\limits_{u\in\mathcal{U}_k(x)}\underset{(w_k|x_k=x,u_k=u)}{E}\left[g_k(x_k,u_k,w_k)+J_{k+1}(f_k(x_k,u_k,w_k))\right]$}

furthermore, if for each $k$ and $x\in\mathcal{S}_k,\ u^\ast=:\mu_k^\ast(x)$ minimizes the recursion equation, the policy $\pi^\ast=(\mu_0^\ast(\cdot),\mu_1^\ast(\cdot),\ldots,\mu_{N-1}^\ast(\cdot))$ is optimal.
\end{theorem}

\subsection{Converting non-standard Problems to the Standard Form}

\subsubsection{Time Lags}

Assume the dynamics have the following form:

\mportant{$x_{k+1}=f_k(x_k,x_{k-1},u_k,u_{k-1},w_k)$}

\begin{itemize}
\item Let $y_k:=x_{k-1},\ s_k:=u_{k-1}$ and the augmented state vector $\tilde{x}_k:=(x_k,y_k,s_k)$.
\item The dynamics of the augmented state then become

\mportant{$\tilde{x}_{k+}=\begin{bmatrix}
x_{k+1}\\y_{k+1}\\s_{k+1}
\end{bmatrix}=\begin{bmatrix}
f_k(x_k,y_k,u_k,s_k,w_k)\\x_k\\u_k
\end{bmatrix}=:\tilde{f}_k(\tilde{x}_k,u_k,\omega_k)$}
\end{itemize}

\subsubsection{Correlated Disturbances}

If the disturbance is correlated across time (colored noise) it can be modelled as follows:

\begin{align*}
w_k&=C_ky_{k+1}\\
y_{k+1}&=A_ky_k+\xi_k
\end{align*}

where $A_k,C_k$ are given and $\xi_k,\ k=0,\ldots,N-1$ are independent random variables.

\begin{itemize}
\item Let the augmented state vector $\tilde{x}_k:=(x_k,y_k)$. Note that now $y_k$ must be observed at time $k$, which can be done using a state estimator.
\item $y_k$ is an internal state of the filter that generates the noise realization $w_k$
\item The dynamics of the augmented sate vector then become:

\mportant{$\tilde{x}_{k+1}=\begin{bmatrix}
x_{k+1}\\y_{k+1}
\end{bmatrix}=
\begin{bmatrix}
f_k(x_k,u_k,C_k(A_ky_k+\xi_k))\\A_ky_k+\xi_k
\end{bmatrix}
=:\tilde{f}_k(\tilde{x}_k,u_k,\xi_k)$}

\item When augmenting the state, the cost function becomes increasingly complex $\rightarrow$ \textbf{curse of dimensionality}!
\end{itemize}

\subsubsection{Forecasts}

Each time period has its own forecast that reveals the probability distribution of $w_k$ and possibly of future disturbances.

At the beginning of each period $k$, we receive a prediction $y_k$ that $w_k$ will attain a probability distribution out of a given finite collection of distributions $\{p_{w_k|y_k}(\cdot|1),p_{w_k|y_k}(\cdot|2),\ldots,p_{w_k|y_k}(\cdot|m)\}$. In particular, we receive a forecast that $y_k=i$ and thus $p_{w_k|x_k}(\cdot,|i)$ is used to generate $w_k$. Furthermore the forecast itself has a given a priori probability distribution, namely

\mportant{$y_{k+1}=\xi_k$}

where $\xi_k$ are independent random variables taking value $i\in\{1,2,\ldots,m\}$ with probability $p_{\xi_k}(i)$.

\begin{itemize}
\item Let the augmented state vector $\tilde{x}_k=(x_k,y_k)$. Since the forecast $y_k$ is known at time $k$ we still have perfect information.
\item We defined our new disturbance as $\tilde{w}_k:=(w_k,\xi_k)$ with the distribution

\begin{align*}
p(\tilde{w}_k|\tilde{x}_k,u_k)&=p(w_k,\xi_k|x_k,y_k,u_k)\\
&=p(w_k|x_k,y_k,u_k,\xi_k)p(\xi_k|x_k,y_k,u_k)\\
&=p(w_k|x_k)p(\xi_k)
\end{align*}
\item The dynamics become

\mportant{$\tilde{x}_{k+1}=\begin{bmatrix}
x_{k+1}\\y_{k+1}
\end{bmatrix}=\begin{bmatrix}
f_k(x_k,u_k,w_k)\\\xi_k
\end{bmatrix}=:\tilde{f}_k(\tilde{x}_k,u_k,\tilde{w}_k)$}
\item The DPA becomes:

\textbf{Initialization}

\mportant{$J_N(\tilde{x})=J_N(x,y)=g_N(x),\quad x\in S_N,y\in\{1,\ldots,m\}$}

\textbf{Recursion}

\important{$\begin{matrix}J_k(\tilde{x})=J_k(x,y)=\underset{u\in\mathcal{U}_k(x_k)}{\min}\underset{(w_k|y_k=y)}{E}\left[g_k(x,u,w_k)+\sum\limits_{i=1}^mp_{\xi_k}(i)J_{k+1}(f_k(x,u,w_k),i)\right]\\\ \forall x\in \mathcal{S}_k,\ \forall y\in\{1,\ldots,m\},\forall k=N-1,\ldots,0\end{matrix}$}
\end{itemize}

\section{Infinite Horizon Problem}

We are dealing with a linear time-invariant system:

\mportant{$x_{k+1}=f(x_k,u_k,w_k),\quad x_k\in\mathcal{S},\ u_k\in\mathcal{U}(x_k),\ w_k\sim p_{w|x,u},\ k=0,\ldots,N-1$}

The control inputs are generated by an admissible policy $\pi\in\Pi$:

\mportant{$\pi=\left(\mu_0(\cdot),\mu_1(\cdot),\ldots,\mu_{N-1}(\cdot)\right)$}

such that

\mportant{$\mu_k=\mu_k(x_k),\ u_k\in\mathcal{U}(x_k),\ \forall\ \in\ \mathcal{S},\ \forall k$}

The cost is a function of time-invariant stage costs:

\mportant{$\sum\limits_{k=}^{N-1}g(x_k,u_k,w_k)$}

The expected closed loop cost of starting at $x$ associated with policy $\pi\in\Pi$:

\mportant{$J_\pi(x)=\underset{(X_1,W_0|x_0=x)}{E}\left[\sum\limits_{k=0}^{N-1}g(x_k,\mu_k(x_k),w_k)\right]$}

\textbf{Objective:}

\mportant{$\pi^\ast=\underset{\pi\in\Pi}{\text{arg min}}J_\pi(x)$}

Analyse behaviour when $N\rightarrow\infty$:

\begin{align*}
J_N(x)&=0\quad \forall x\in\mathcal{S}\\
J_k(x)&=\underset{u\in\mathcal{U}(x)}{\min}\underset{(w|x=x,u=u)}{E}[g(x,u,w)+J_{k+1}(f(x,u,w))],\quad \forall x\in\mathcal{S},\quad \forall k=N-1,\ldots,0
\end{align*}

By index substitution: $l:=N-k$ and $V_l(\cdot):=J_{N-l}(\cdot)$:

\begin{align*}
V_0(x)&=0\quad \forall x\in\mathcal{S}\\
V_l(x)&=\underset{u\in\mathcal{U}(x)}{\min}\underset{(w|x=x,u=u}{E}[g(x,u,w)+V_{l-1}(f(x,u,w))],\quad\forall x\in\mathcal{S},\quad\forall l=1,\ldots,N
\end{align*}

Now assume that for each $x\in\mathcal{S}$ the sequence $V_l(x)$ approaches a certain value as $N\rightarrow\infty$:

\importname{Bellman Equation (BE)}{$J(x)=\underset{u\in\mathcal{U}(x)}{\min}\underset{(w|x=x,u=u)}{E}[g(x,u,w)+V_{l-1}(f(x,u,w))],\forall x\in\mathcal{S}$}

\subsection{The Stochastic Shortest Path (SSP) Problem}

Consider a finite state, time-invariant system:

\begin{align*}
x_{k+1}=w_k,\quad x_k\in\mathcal{S}\\
Pr(w_k=j|x_k=i,u_k=u)=P_{ij}(u),\quad u\in\mathcal{U}(i)
\end{align*}

The expected closed loop cost of starting at $i$ associated with policy $\pi$ becomes:

\mportant{$J_\pi(i)=\underset{X_1,W_0|x_0=i)}{E}\left[\sum\limits_{k=0}^{N-1}g(x_k,\mu_k(x_k),w_k)\right]$}

We \textbf{assume} that there exists a cost-free termination state, which we designate as state 0. In praticular, there are $n+1$ states with $\mathcal{S}=\{0,1,\ldots,n\}$ where

\mportant{$P_{00}(u)=1$ and $g(0,u,0)=0,\quad \forall u\in\mathcal{U}(0)$}

The objective is then:

\mportant{$\pi^\ast=\underset{\pi\in\Pi}{\text{arg min}}J_\pi(i)$}

\begin{define}
A policy is \textbf{stationary} if it is the same for all times such that $\pi = (\mu(\cdot),\mu(\cdot),\ldots,\mu(\cdot))$ which is written just as $\mu$.
\end{define}

\begin{define}
A stationary policy is said to be \textbf{proper} if, when using this policy there exists an integer $m$ such that

\mportant{$Pr(x_m=0|x_0=i)>0$}

If a policy is not proper it is said to be \textbf{improper}.
\end{define}

Further it is \textbf{assumed} that there exists at least one proper policy $\mu\in\Pi$. Furthermore, for every improper policy $\mu'$, the corresponding cost function $J_{\mu'}$ is infinity for at least one state $i\in\mathcal{S}$.

Based on that it can be proven that the final state will be reached with probability 1:

\begin{align*}
Pr(x_m=0|x_0=i)&=\alpha >0\\
Pr(x_m\neq 0|x_0=i)&=1-\alpha,\ i\in\mathcal{S}\textbackslash\{0\}\\
Pr(x_{2m}\neq 0|x_0=i)&=Pr(x_{2m}\neq 0,x_m\neq 0|x_0=i)\\
&=\underbrace{Pr(x_{2m}\neq 0|x_m\neq 0,x_0=i)}_{1-\alpha}\underbrace{Pr(x_m\neq 0|x_0=i)}_{1-\alpha}\\
\Rightarrow Pr(x_{2m}=0|x_0=i)&=1-(1-\alpha)^2\\
&\Rightarrow \lim\limits_{N\rightarrow \infty}Pr(x_N=0|x_0=1)=1
\end{align*}

\subsection{Theorem 4.1}

\begin{enumerate}
\item Given any initial conditions $V_0(1),\ldots,V_0(n)$, the sequence $V_l(i)$ generated by the iteration

\mportant{$V_{l+1}(i)=\underset{u\in\mathcal{U}}{\min}\left(q(i,u)+\sum\limits_{j=1}^nP_{ij}(u)V_l(j)\right),\quad \forall i\in\mathcal{S}\textbackslash\{0\}$}

where

\mportant{$q(i,u):=\underset{(x|x=i,u=u)}[g(x,u,w)]$}

\item The optimal costs satisfy the Bellman Equation:

\mportant{$J^\ast(i)=\underset{u\in\mathcal{U}}{\min}\left(q(i,u)+\sum\limits_{j=1}^nP_{ij}(u)J^\ast(j)\right)\forall i\in\mathcal{S}\textbackslash\{0\}$}

\item The solution to the BE is unique
\item The minimizing $u$ for each $i\in\mathcal{S}\textbackslash\{0\}$ of the BE gives an optimal policy, which is proper.
\end{enumerate}

See lecture 4 for an intuition why this is true.

\end{multicols*}

\end{document}